% Template:     Informe LaTeX
% Documento:    Archivo principal
% Versión:      7.1.9 (22/04/2021)
% Codificación: UTF-8
%
% Autor: Pablo Pizarro R.
%        Facultad de Ciencias Físicas y Matemáticas
%        Universidad de Chile
%        pablo@ppizarror.com
%
% Manual template: [https://latex.ppizarror.com/informe]
% Licencia MIT:    [https://opensource.org/licenses/MIT]

% CREACIÓN DEL DOCUMENTO
\documentclass[letterpaper,oneside]{article}

% INFORMACIÓN DEL DOCUMENTO
\def\titulodelinforme {Informe de Interferometría}
\def\temaatratar {}

\def\autordeldocumento {Vicente Maldonado}
\def\nombredelcurso {Astronomía Experimental}
\def\codigodelcurso {AS3201-2}

\def\nombreuniversidad {Universidad de Chile}
\def\nombrefacultad {Facultad de Ciencias Físicas y Matemáticas}
\def\departamentouniversidad {Departamento de Astronomía}
\def\imagendepartamento {departamentos/das}
\def\imagendepartamentoparams {height=1.57cm}
\def\localizacionuniversidad {Santiago, Chile}

% INTEGRANTES, PROFESORES Y FECHAS
\def\tablaintegrantes {
\begin{tabular}{ll}
	Autor:
	& \begin{tabular}[t]{l}
		Vicente Maldonado A. \\
	\end{tabular} \\
	Profesor:
	& \begin{tabular}[t]{l}
		Ricardo Finger C.
	\end{tabular} \\
	Auxiliar:
	& \begin{tabular}[t]{l}
		Javier Silva F.
	\end{tabular} \\
	\multicolumn{2}{l}{Fecha de entrega: \today} \\
\end{tabular}}{
}

% IMPORTACIÓN DEL TEMPLATE
\input{template}

% INICIO DE PÁGINAS
\begin{document}
	
% PORTADA
\templatePortrait

% CONFIGURACIÓN DE PÁGINA Y ENCABEZADOS
\templatePagecfg

% CONFIGURACIONES FINALES
\templateFinalcfg

% ======================= INICIO DEL DOCUMENTO =======================

\section{Introducción}
En este informe se estudiará la técnica de interferometría, que consiste en un arreglo de una cierta cantidad de antenas para observar un mismo objeto, logrando un mejor detalle angular. El objetivo de este estudio es, a partir de datos obtenidos del interferómetro del Departamento de Ingeniería eléctrica de la FCFM, calcular el diámetro angular del Sol.\\

El interferómetro cuenta con dos antenas de $2.3$ m de diámetro separadas $B = 37.85$ $(\pm 0.1)$ m en la dirección Este-Oeste  y $\alpha = 85.2$ ($\pm 0.3$) grados sexagesimales respecto al norte geográfico. Las antenas obtuvieron datos del Sol entre las $14:40$ hrs y $23:19$ y al día siguiente entre las $09:38$ hrs y las $15:12$ hrs, obtuvieron espectros para la temperatura de antena $T [K]$ de una longitud de onda $\lambda$ de $21.1$ cm. Inicialmente, se separaron los datos correspondientes a la mañana y el atardecer y se promediaron los canales obteniendo una única temperatura de antena por medición. Luego, se filtró el ruido de la temperatura de antena y posteriormente se elimino el offset de las mediciones en el espacio de Fourier.\\

En paralelo, se obtuvo una expresión para la apertura $B_s$, que es la proyección de $B$ en el frente de onda plano, en función del azimut y elevación de la fuente.\\

Finalmente, se realizó un ajuste de los datos obtenidos con el modelo de visibilidad del sol con un interferómetro de dos antenas, obtenido integrando el patrón de antena en la forma del disco solar, una vez realizado el ajuste se obtendrá el diámetro angular del Sol. El resultado obtenido para el diámetro angular del Sol fue de \textbf{0.46°} en la mañana y de \textbf{0.62°} en la Tarde.

\section{Marco Teórico}
\begin{figure}
  \centering
  \includegraphics[height=5cm]{../graficos/interferometro.png}
  \caption{Esquema de la ubicación de las antenas, se observa que se encuentran a una distancia $B$ mientras que la apertura corresponde a $B_s$.}
\end{figure}

A partir de la Figura 1 es posible encontrar que la apertura $B_s$ viene dada por la siguiente expresión:

\begin{equation}
    B_s = B\sqrt{1-cos^2(az-\alpha)\cdot cos^2(el)}
\end{equation}

En donde $B$ es la separación de las antenas ($37.85$ m), $az$ con $el$ son el azimut y la elevación respectivamente, medida en radianes y $\alpha$ es el ángulo respecto al norte geográfico. La apertura para los datos de la mañana y tarde se observan en la Figura 2.\\

\begin{figure}
  \centering
  \includegraphics[height=7cm]{../graficos/aperturas.png}
  \caption{Apertura $B_s$ de cada medición.}
\end{figure}

La función de visibilidad, se obtiene integrando el patrón de radiación $P(\Omega)$ y el brillo $I(\Omega)$, considerando al sol como un disco circular.

$$V = \int \int I(\Omega)P(\Omega)d\Omega$$

Se integra con un brillo constante $I(\Omega)=I$ y un patrón de radiación $P(\Omega)=cos(\frac{2\pi B}{\lambda}cos(\theta))$, y parametrizando el disco solar como $x^2 + y^2 = R^2$, con $R$  el radio angular del Sol. El resultado para la visibilidad, en función del diámetro angular del Sol $(d)$, viene dado por la siguiente ecuación:

\begin{equation}
    V(d) = \left |\frac{sen(\pi d B_s)}{\pi d B_s} \right |
\end{equation}

\section{Desarrollo}
El interferómetro entrego los datos correspondientes a la mañana y tarde, se promediaron los canales para obtener una única temperatura de antena para cada medición, los datos sin procesar se observan en la Figura 3.\\

\begin{figure}
  \centering
  \includegraphics[height=7cm]{../graficos/rawdata.png}
  \caption{Datos sin filtrar para la temperatura de antena de cada medición.}
\end{figure}

Los datos de los extremos corresponden al inicio del día y la medianoche, que tienen mucha interferencia producto del ruido de la ciudad, por esta razón fueron eliminados, también se filtraron todos los datos tal que su diferencia con el promedio (RMS) fuera mayor a $0.5$ veces la raíz media cuadrada. Los datos luego de este filtro se observan en la Figura 4.\\

\begin{figure}
  \centering
  \includegraphics[height=7cm]{../graficos/primerfiltro.png}
  \caption{Datos de la temperatura de antena, luego de filtrar el amanecer y el anochecer y aplicar un filtro a las temperaturas altas.}
\end{figure}

A continuación, se realizó la transformada de Fourier para obtener los espectros, los cuales fueron filtrados, anulando las frecuencias menores a $0.003$ Hz y mayores a $0.04$ Hz, eliminando de este modo el DC Offset y el ruido a alta frecuencia.

\section{Resultados}
El espectro y los datos luego de anular las frecuencias ruidosas, se observan en las Figuras 5 y 6 respectivamente.

\begin{figure}
  \centering
  \includegraphics[height=7cm]{../graficos/cleanespectros.png}
  \caption{Espectro para la temperatura de antena.}
\end{figure}

\begin{figure}
  \centering
  \includegraphics[height=7cm]{../graficos/cleandata.png}
  \caption{Datos filtrados de la temperatura de antena, para cada medición.}
\end{figure}

Una vez se obtienen los datos filtrados, es posible calcular la apertura a partir de $B$, el azimut y la elevación, tal como se mostró en la Figura 2, con estos datos, se gráfica el valor absoluto de la temperatura de antena en función de la apertura, junto con el ajuste realizado a la función de visibilidad, como se observa en la Figura 7.

\begin{figure}
  \centering
  \includegraphics[height=7cm]{../graficos/ajuste.png}
  \caption{Ajuste de la función visibilidad (Naranja) y el valor absoluto de la temperatura de antena (Azul).}
\end{figure}

Los parámetros del ajuste indicaron que el diámetro angular del Sol es de \textbf{0.46°} en la mañana y de \textbf{0.62°} en la tarde, considerando una amplitud de la función de visibilidad de $2500$ K. El promedio de estos dos valores entrega un diámetro promedio de \textbf{0.54°}.

\section{Análisis y Conclusiones}
El valor obtenido para el tamaño angular del Sol correspondía parcialmente al indiciado por la literatura, de 0.53°, muy cercano al promedio obtenido. Sin embargo, la variación del tamaño entre la mañana en la tarde es muy elevado (casi un $30\%$). Esto se puede deber principalmente a los errores asociados a la contaminación electromagnética en Santiago, existiendo muchas fuentes de ruido ya sean señales de radio , televisión, internet, etc. También es posible que durante el proceso de filtrado de datos se haya perdido mucha información.\\

$$\theta_{arc} = 2 \frac{\lambda [cm]}{D [m]}$$

Por otro lado, el diámetro angular del Sol comparado con el main beam, calculado con la expresión anterior y considerando el diámetro de la antena ($2.3$ m), para una sola antena es de $\theta_{arc} = 18347.8$ ($5.09$°), demasiado alto, sin embargo, al considerar ambas antenas mediante la interferometría, $\theta_{arc} = 1145$ ($0.3$°) con una distancia $B$ de $37.85$ m, se logra un número mucho mas bajo y es debido a esto que la técnica de observación fue efectiva.\\

Una de las posibles limitantes del interferómetro es que la función de visibilidad decae demasiado rápido y eso hizo que los datos no se ajustaran perfectamente al modelo, pero considerando que tener dos antenas de $2.3$ m es muchísimo más sencillo en términos tecnológicos y humanos que una sola antena de $37.5$ m, hace que la interferometría sea una alternativa viable y a la vez precisa.\\

En conclusión, la interferometría es una técnica que permite obtener datos precisos utilizando una solida base teórica, tal como fue demostrado en el ajuste del modelo y calculo del diámetro angular del Sol.\\


\section{Anexos}
\begin{itemize}
  \item \href{https://github.com/vecheto/astro_experimental/tree/main/Tarea\%204}{Repositorio en GitHub}
  \item \href{https://www.matem.unam.mx/~barot/clases/2012-2/24medicion.pdf}{Tamaños y distancias celestes}
\end{itemize} 
 
\end{document}
